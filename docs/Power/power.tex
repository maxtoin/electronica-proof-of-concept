%Copyright 2014 Jean-Philippe Eisenbarth
% Modified by Ivan Rodriguez <alu0100765755@ull.edu.es> 2018
%This program is free software: you can 
%redistribute it and/or modify it under the terms of the GNU General Public 
%License as published by the Free Software Foundation, either version 3 of the 
%License, or (at your option) any later version.
%This program is distributed in the hope that it will be useful,but WITHOUT ANY 
%WARRANTY; without even the implied warranty of MERCHANTABILITY or FITNESS FOR A 
%PARTICULAR PURPOSE. See the GNU General Public License for more details.
%You should have received a copy of the GNU General Public License along with 
%this program.  If not, see <http://www.gnu.org/licenses/>.

%Based on the code of Yiannis Lazarides
%http://tex.stackexchange.com/questions/42602/software-requirements-specification-with-latex
%http://tex.stackexchange.com/users/963/yiannis-lazarides
%Also based on the template of Karl E. Wiegers
%http://www.se.rit.edu/~emad/teaching/slides/srs_template_sep14.pdf
%http://karlwiegers.com
\documentclass{scrreprt}
\usepackage{listings}
\usepackage{underscore}
\usepackage{mwe} % for blindtext and example-image-a in example
\usepackage{wrapfig}
\usepackage[bookmarks=true]{hyperref}
\usepackage[utf8]{inputenc}
\usepackage[spanish]{babel}

% BIBLIOGRAFÍA
\usepackage[backend=biber,citestyle=numeric]{biblatex}
\bibliography{libreria}

% Figuras
\usepackage{float}
\usepackage{graphicx}
\usepackage{subfig}

%Colores
% \usepackage{color}
\usepackage{xcolor}
\hypersetup{
    colorlinks,
    linkcolor={red!50!black},
    citecolor={blue!50!black},
    urlcolor={blue!80!black}
}

\definecolor{gris10}{rgb}{0.9,0.9,0.9}

\usepackage{listings}

\definecolor{mGreen}{rgb}{0,0.6,0}
\definecolor{mGray}{rgb}{0.5,0.5,0.5}
\definecolor{mPurple}{rgb}{0.58,0,0.82}
\definecolor{backgroundColour}{rgb}{0.95,0.95,0.92}


\hypersetup{
    bookmarks=false,    % show bookmarks bar?
    pdftitle={Especificaciones de potencia para la prueba de concepto de comunicaciones ópticas},    % title
    pdfauthor={Iván Rodríguez - TeideSat},                     % author
    pdfsubject={TeX and LaTeX},                        % subject of the document
    pdfkeywords={TeX, LaTeX, graphics, images}, % list of keywords
    colorlinks=true,       % false: boxed links; true: colored links
    linkcolor=blue,       % color of internal links
    citecolor=black,       % color of links to bibliography
    filecolor=black,        % color of file links
    urlcolor=purple,        % color of external links
    linktoc=page            % only page is linked
}%
\def\myversion{0.1 }
\date{}
%\title
\usepackage{hyperref}
\begin{document}
 \hfill\includegraphics[width=2.7cm,height=2.6cm]{Logo.png}\\[\bigskipamount]\vspace*{-1.5cm}
\begin{flushright}
    \rule{15cm}{3pt}\vskip0.8cm
    \begin{bfseries}
        \Huge{ESPECIFICACIONES \\ DE POTENCIA}\\
        \vspace{1cm}
        para\\
        \vspace{1cm}
        Prueba de concepto de comunicaciones ópticas\\
        \vspace{2.5cm}
        \LARGE{Versión \myversion aprobada}\\
        \vspace{3cm}
        \large{Escrito por  Sección Electrónica} \\
        \vspace{0.5cm}
        TeideSAT\\
        \vspace{1.5cm}
        \today\\
    \end{bfseries}
\end{flushright}

\tableofcontents


\chapter*{Revisión histórica de cambios}

\begin{center}
    \begin{tabular}{|c|c|c|c|}
        \hline
	    Nombre & Fecha & Razón de cambios & Versión\\
        \hline
	    Iván Rodríguez & 30/11/2018 & Comienzo de la redacción & 0.1\\
        \hline
    \end{tabular}
\end{center}

\chapter{Introducción}

\section{Propósito}

El propósito de los \textit{dummies} para el proyecto de prueba de concepto de comunicaciones ópticas es la verificación en tierra de la posibilidad de envío y recepción de un mensaje codificado. Es por ello necesario, establecer las especificaciones de los prototipos para su estudio en detalle.

\section{Convención de documentos}
Para la redacción de este documento se han seguido las directrices indicadas en el libro \textbf{Low Earth Orbit Satellite Design}~\cite{sebestyenlow2018}. Se han seguido las directrices de este documento para los cálculos de las especificaciones.

\section{Audiencia destinataria y sugerencias de lectura}
Este documento está dirigido principalmente a los miembros implicados en el desarrollo de las estaciones para los test basados en comunicaciones ópticas. En el resto del documento pueden encontrar las siguientes secciones:
\begin{itemize}
 \item En el capítulo 1...
 \item En el capítulo 2...
\end{itemize}

La lectura de los capítulos X e Y son de especial interés para desarrolladores.

\section{Alcance del documento}
En este proyecto, se especifican en detalle las especificaciones y requerimientos de potencia tanto para la estación emisora como para la estación receptora. Se justificarán todos los cálculos realizados para la elaboración de las tablas de potencia.

\section{Referencias}

La documentación utilizada para la redacción de este documento ha sido la siguiente:

\printbibliography[heading=none]

\chapter{Especificaciones de potencia para el receptor}

\chapter{Especificaciones de potencia para el emisor}

\end{document}
